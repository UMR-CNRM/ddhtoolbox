%
%-----------------------------------------------------------------------
% Têtes et pieds de pages plus stylées.
%-----------------------------------------------------------------------
%
\usepackage{fancyhdr}
%
%-----------------------------------------------------------------------
% Jeu de caractères.
%-----------------------------------------------------------------------
%
\usepackage[latin1]{inputenc}
%
%-----------------------------------------------
% Fontes.
%-----------------------------------------------
%
\usepackage[LGR,T1]{fontenc}
%\usepackage[T1]{fontenc}
%
%-----------------------------------------------
% Graphiques.
%-----------------------------------------------
%
\usepackage[usenames,dvipsnames]{color}
\usepackage{graphicx}
%
%-----------------------------------------------
% Interface HTML.
%-----------------------------------------------
%
\usepackage{html,htmllist}
%
%-----------------------------------------------
% Gestion des références bibliographiques.
%-----------------------------------------------
%
\usepackage{apalike}
%
%-----------------------------------------------
% Choix de la langue.
% Actuellement il y a une incompatibilité pour l'"euro"!...
% - En effet si on opte pour le paquet "[greek,frenchb]{babel}" on a bien le signe "euro" par \euros,
%   mais on a le sommaire en Anglais ("Contents"), de même pour les références ("References").
% - Si on opte pour le paquet "{babel}" on a tout bien en Français
%   mais on n'a plus le signe "euro" par \euro!...
%-----------------------------------------------
%
%\usepackage[greek,frenchb]{babel}
\usepackage{babel}

