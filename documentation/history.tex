\chapter{History}

\null
\vspace{1.0cm}
\begin{description}
  \item [1991:] Initial analysis and coding of the {\tt DDH} software (Alain Joly).
  \item [1992:] Introduction of entropy and kinetic energy budgets (Martin Janousek, Jean-Marcel Piriou).
  \item [1993:] Introduction of relative humidity, liquid and ice water (Jean-Marcel Piriou).
  \item [1997:] New surface fields, IFS model (Pedro Viterbo).
  \item [2000:] Diagnose surface "tiles", IFS model (Christian Jakob).
  \item [2006:] Extract AROME physical data flow, interface to DDH routines (Tomislav Kovacic).
  \item [2006:] Write AROME DDH documentation (Tomislav Kovacic).
  \item [2007-11:] Rewrite the budget tool "ddhb",
    still based on the "ddhi" and "ddht" existing ones (Alex Deckmyn, Jean-Marcel Piriou, Tomas Kral).
  \item [2008-07:] Draft translation of the French DDH documentation into English,
    by Jean Maziejewski International Sekretarski, approved by Jean-Marcel Piriou.
  \item [2008-07:] Create the ddhtoolbox, write its documentation (Jean-Marcel Piriou).
  \item [2009:] Interface AROME physics with DDH, new flexible dataflow 
    for AROME, ALADIN and ARPEGE (Olivier Rivière).
  \item [2018:] Dynamical DDH: semi-lagrangian advection, horizontal diffusion, semi-implicit (Fabrice Voitus).
  \item [2019:] Flexible DDH: adding a new field in calling NEW\_ADD\_FIELD\_3D (Fabrice Voitus).
  \item [2023:] A new section was written in the ddhtoolbox documentation, to learn step by step, from examples, the use of the DDH tools (ddhi, ddhr, ddht, etc) (Jean-Marcel Piriou). 
  \item [2023:] The dd2gr plot tool is now part of the ddhtoolbox, to allow an easier start with DDH use (Jean-Marcel Piriou).
\end{description}
