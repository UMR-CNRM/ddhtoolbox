%
%-----------------------------------------------------------------------
% Choix de la fonte "times", plus lisible lorsque convertie en PDF.
%-----------------------------------------------------------------------
%
\usepackage{times}
%
%---------------------------------------------------------------------------
%
% Commandes communes aux deux langues et aux deux styles.
%
%---------------------------------------------------------------------------
%
% Caractères spéciaux.
%
% Début de paragraphe après ligne section ou chapitre.
%\newcommand{\pa}{}
\newcommand{\pa}{\noi}
%
% Début de paragraphe au sein d'une section ou chapitre.
%\newcommand{\p}{\vspace{0.4cm}}
\newcommand{\p}{\vspace{0.4cm}\noi}
\newcommand{\xx}{\p [...]}
\newcommand{\refp}[1]{(\ref{#1} p. \pageref{#1})}
\newcommand{\euros}{\textgreek{\euro}}
\newcommand{\lp}{\left(}
\newcommand{\rp}{\right)}
\newcommand{\lb}{\left\{}
\newcommand{\rb}{\right\}}
\newcommand{\lc}{\, \left[ \,}
\newcommand{\rc}{\, \right] \,}
\newcommand{\rd}{\, \right]}
\newcommand{\mb}{\null}
%\newcommand{\up}[1]{\raisebox{1ex}{\footnotesize#1}}
\newcommand{\dpdp}[2]{\frac{\partial #1}{\partial #2}}
\newcommand{\dd}[2]{\frac{d #1}{d #2}}
\newcommand{\ta}{{\theta}}
\newcommand{\tal}{{\theta_L}}
\newcommand{\tav}{{\theta_V}}
\newcommand{\tavl}{{\theta_{VL}}}
\newcommand{\tapw}{{\theta'_{w}}}
\newcommand{\tae}{{\theta_{E}}}
\newcommand{\taes}{{\theta_{ES}}}
\newcommand{\la}{\lambda}
\newcommand{\vp}{\varphi}
\newcommand{\ve}{\varepsilon}
\newcommand{\vt}{\vartheta}
%
\renewcommand{\ss}{\scriptstyle}
\newcommand{\sss}{\scriptscriptstyle}
\newcommand{\ds}{\displaystyle}
%
\newcommand{\noi}{\noindent}
\newcommand{\hs}{\hspace{1.5em}}
\newcommand{\ms}{\medskip}
%
\newcommand{\bey}{\begin{eqnarray}}
\newcommand{\eey}{\end{eqnarray}}
\newcommand{\bez}{\begin{eqnarray*}}
\newcommand{\eez}{\end{eqnarray*}}
\newcommand{\bay}{\begin{array}}
\newcommand{\eay}{\end{array}}
%
\newcommand{\ovl}[1]{\mkern1mu\overline{\mkern-1mu#1\mkern-1mu}\mkern1mu}
\newcommand{\ova}{\overrightarrow}
%
%-----------------------------------------------------------------------
% Ecrire du texte au sein d'une formule.
%-----------------------------------------------------------------------
%
\newcommand{\texte}[1]{\mathop{\rm #1}\nolimits}
%
% Trigonométrie et opérateurs.
\newcommand{\rot}{\mathop{\rm \overrightarrow{\vphantom{i}\rm ro}\mkern-.5mu t}\nolimits}
%\newcommand{\grad}{\mathop{\rm \overrightarrow{\vphantom{i}\rm gra}d}\nolimits}
\newcommand{\grad}{\vec{\nabla}}
\newcommand{\Grad}{\mathop{\rm \bf grad}\nolimits}
\newcommand{\Div}{\mathop{\rm div}\nolimits}
\renewcommand{\div}{\mathop{\rm div}\nolimits}
\newcommand{\Arctan}{\mathop{\rm Arctan}\nolimits}
\newcommand{\Arccos}{\mathop{\rm Arccos}\nolimits}
\newcommand{\Arcsin}{\mathop{\rm Arcsin}\nolimits}
\newcommand{\cotan}{\mathop{\rm cotan}\nolimits}
\newcommand{\ch}{\mathop{\rm ch}\nolimits}
\newcommand{\sh}{\mathop{\rm sh}\nolimits}
%\newcommand{\th}{\mathop{\rm th}\nolimits}
%
% Intégrales.
%\newcommand{\iint}{\int\!\!\!\int}
%\newcommand{\iiint}{\int\!\!\!\int\!\!\!\int}
%
% Date.
\newcommand{\jjmmaa}{\number\day.\number\month.\number\year}
%
% Fonctions texte.
\newcommand{\ie}{{\em i.e.}}
\newcommand{\cf}{{\em cf}}
%
% Fonctions mathématiques.
\newcommand{\AdvU}{(\vec{u}\cdot\vec{\nabla})}
\newcommand{\AdvV}{(\vec{v}\cdot\vec{\nabla})}
\newcommand{\DerDN}[3]{\frac{\ds d^{#3} #1}{\ds d #2^{#3}}}
\newcommand{\DerPN}[3]{\frac{\ds \partial^{#3} #1}{\ds \partial #2^{#3}}}
\newcommand{\DerP}[2]{\frac{\ds \partial #1}{\ds \partial #2}}
\newcommand{\DerD}[2]{\frac{\ds d #1}{\ds d #2}}
\newcommand{\us}[1]{\frac{1}{#1}}
\newcommand{\VarGen}{\mu}
\newcommand{\ee}[1]{\cdot 10^{#1}}
%
% Unités.
\newcommand{\wm}{W\,m^{-2}}
%
% Fonctions ARPEGE.
\newcommand{\sbou}{{\sc Conv-Arpege-PNT}}
\newcommand{\sjef}{{\sc Conv-Arpege-Climat}}
\newcommand{\dpsg}{\frac{\ds dp}{\ds g}}
\newcommand{\mf}{{\sc Meteo-France}}
\newcommand{\ARP}{{\sc Arpege}}
\newcommand{\arp}{{\sc Arpege}}
\newcommand{\arpala}{{\sc Arpege-Aladin}}
\newcommand{\arpalaaro}{{\sc Arpege-Aladin-Arome}}
\newcommand{\alaro}{{\sc Alaro}}
\newcommand{\aro}{{\sc Arome}}
\newcommand{\ARO}{{\sc Arome}}
\newcommand{\arome}{{\sc Arome}}
\newcommand{\arptro}{{\sc Arpege-Tropiques}}
\newcommand{\mnh}{{\sc M\'eso-NH}}
\newcommand{\cnh}{{\sc COME-NH}}
\newcommand{\eme}{{\sc Emeraude}}
\newcommand{\ala}{{\sc Aladin}}
\newcommand{\ALA}{{\sc Aladin}}
\newcommand{\alanh}{{\sc Aladin-NH}}
\newcommand{\alae}{{\sc Aladin-Europe Centrale}}
\newcommand{\alaf}{{\sc Aladin-France}}
\newcommand{\alam}{{\sc Aladin-Maroc}}
\newcommand{\arppnt}{{\sc Arpege-PNT}}
\newcommand{\arpc}{{\sc Arpege-Climat}}
\newcommand{\ARPC}{{\sc Arpege-Climat}}
\newcommand{\arpifs}{{\sc Arpege-Ifs}}
\newcommand{\ifs}{{\sc Ifs}}
\newcommand{\arpa}{{\sc Arpege-Aladin}}
\newcommand{\mesonh}{{\sc Méso-NH}}
\newcommand{\cep}{{\sc CEP}}
\newcommand{\ecmwf}{{\sc Ecmwf}}
\newcommand{\grib}{{\sc GRIB}}
\newcommand{\GRIB}{{\sc GRIB}}
\newcommand{\re}{r_{\eta}}
\newcommand{\fp}[1]{F_{p#1}}
\newcommand{\cp}[1]{{c_{p}}_{#1}}
\newcommand{\Cp}[1]{{C_{p}}_{#1}}
\newcommand{\divi}[1]{\div_{#1}}
\newcommand{\interv}[2]{\lc #1,\,#2\rc}
\newcommand{\paire}[2]{\lb #1,\,#2\rb}
\newcommand{\ensemble}[2]{\lb #1,\,..., \, #2\rb}
%
\newcommand{\fpcl}[1]{F_{p#1}^{conv-l}}
\newcommand{\fpcn}[1]{F_{p#1}^{conv-n}}
\newcommand{\fpsl}[1]{F_{p#1}^{stra-l}}
\newcommand{\fpsn}[1]{F_{p#1}^{stra-n}}
\newcommand{\fpl}[1]{F_{p#1}^{l}}
\newcommand{\fpn}[1]{F_{p#1}^{n}}
\newcommand{\fccl}[1]{F_{c#1}^{conv-l}}
\newcommand{\fccn}[1]{F_{c#1}^{conv-n}}
\newcommand{\fcsl}[1]{F_{c#1}^{stra-l}}
\newcommand{\fcsn}[1]{F_{c#1}^{stra-n}}
\newcommand{\fcl}[1]{F_{c#1}^{l}}
\newcommand{\fcn}[1]{F_{c#1}^{n}}
\newcommand{\fc}[1]{F_{c#1}}
%
\newcommand{\fcptpcl}{F_{{c_p T}_{prec}}^{conv-l}}
\newcommand{\fcptpcn}{F_{{c_p T}_{prec}}^{conv-n}}
\newcommand{\fcptpsl}{F_{{c_p T}_{prec}}^{stra-l}}
\newcommand{\fcptpsn}{F_{{c_p T}_{prec}}^{stra-n}}
\newcommand{\fcptpl}{F_{{c_p T}_{prec}}^{l}}
\newcommand{\fcptpn}{F_{{c_p T}_{prec}}^{n}}
\newcommand{\fcptp}{F_{{c_p T}_{prec}}}
%
% Convection.
\newcommand{\dlnpb}{(\Delta\ln p)^{b}}
\newcommand{\dlnph}{(\Delta\ln p)^{h}}
\newcommand{\rbtm}{\tilde R_{b}^{-}}
\newcommand{\rbtp}{\tilde R_{b}^{+}}
\newcommand{\rvtp}{\tilde R_{v}^{+}}
\newcommand{\tdconv}[1]{\lp\DerP{#1}{t}\rp_{conv}}
\newcommand{\tdconvp}[1]{\lp\DerP{#1}{t}\rp_{conv\_\,prec}}
\newcommand{\fsd}{F_{s}^{dif\_\,tur}}
\newcommand{\fqd}{F_{q}^{dif\_\,tur}}
\newcommand{\fhd}{F_{cp T}^{dif\_\,tur}}
\newcommand{\tsd}{T_{s}^{dif\_\,tur}}
\newcommand{\tqd}{T_{q}^{dif\_\,tur}}
\newcommand{\thd}{T_{cp T}^{dif\_\,tur}}
\newcommand{\tvasc}{\vec{T}_{\vec{v}}^{conv}}
\newcommand{\omee}{\omega^{*}}
\newcommand{\omec}{\omega_c}
%
%-----------------------------------------------------------------------
% Commande servant à saisir l'essence d'un paragraphe,
% lorsqu'on en est, au cours de la rédaction, encore au stade
% de créer le plan détaillé.
%-----------------------------------------------------------------------
%
\newcommand{\ideepar}[1]{\p[{\bf Paragraphe: }#1]}
%
% Guillemets.
\def\og{\leavevmode\raise.3ex\hbox{$\scriptscriptstyle\langle\!\langle$\kern.05em}}
\def\fg{\leavevmode\unskip\kern.05em\raise.3ex\hbox{$\scriptscriptstyle\rangle\!\rangle$}}
%
% Commande "sujet" ouvrant un nouveau paragraphe séparé du précédent
% par une ligne horizontale et comportant un titre en gras.
\newcommand{\sujet}[1]{\noindent\rule{\textwidth}{0.1mm} {\bf \large #1 \newline }}
%
% Page de titre, 5 arguments:
%   1. Titre.
%   2. Sous-titre.
%   3. Auteur.
%   4. Version et/ou date.
%   5. Nom du fichier graphique à insérer.
%
\newcommand{\entete}[6]{
   {\pagestyle{empty}\null
   \begin{center}
   \begin{tabular}{c}
      \\[1ex] \huge \centerline{#1} \\[1ex]
      \huge #2 \\[2ex]
      \Large #3 \\[2ex] \Large #4 \\[1ex]
   \end{tabular}
   \end{center}
   \null\vspace{1cm}
	\begin{figure}[htbp]
		\centerline{
		\includegraphics
			[angle=0, 
			keepaspectratio=true,
			clip=true,#6]
			{#5}
		}
	\end{figure}
   \newpage\sommaire}}
%
%-----------------------------------------------------------------------
% Graphique commun pour latex et latex2html,
% d'après Ryad El Khatib le 26.5.2003.
%-----------------------------------------------------------------------
%
\newcommand{\graphryad}[5]
	{
	\begin{latexonly}
		\medskip\par
	\end{latexonly}
	\begin{figure}[!h]
		\label{fig:#1}
		\html{\htmlimage{align=center,transparent,antialias}}
		\html{\htmlborder{5}}
		\centering
		\html{\includegraphics[angle=#3]{#1.eps}}
		\latex{\framebox{\includegraphics[scale=#2,angle=#3]{#1.eps}}}
		\begin{latexonly}
			\caption[#4]{#5}
		\end{latexonly}
	\end{figure}
	\html{\begin{center}{\bf Figure~\ref{fig:#1}: }\emph{#5}\end{center}}
	}
%
%-----------------------------------------------------------------------
% Image graphicx en mode figure (ancienne version).
%-----------------------------------------------------------------------
%
\newcommand{\figp}[4]
	{
	\begin{figure}[htbp]
		\centerline{
			\includegraphics
				[angle=#3, 
				width=#2, 
				keepaspectratio=true,
				clip=true]
				{#1}
		}
		\caption{#4}
		\label{#1}
	\end{figure}
	}
%
%-----------------------------------------------------------------------
% Image graphicx en mode figure (nouvelle version).
% [h]: graphique ici.
% [b]: graphique en bas  de page.
% [t]: graphique en haut de page.
% [p]: graphique sur une page à part.
%-----------------------------------------------------------------------
%
\newcommand{\figpn}[6]
	{
	\begin{figure}[htbp]
		\centerline{
			\includegraphics
				[angle=#3, 
				width=#2, 
				keepaspectratio=true,
				clip=true]
				{#1}
		}
		\caption{ {\bf #4} {\em #5} {#6} }
		\label{#1}
	\end{figure}
	}
%
%-----------------------------------------------------------------------
% Image graphicx double. I.e. lit 2 fichiers EPS en entrée,
% et crée une seule figure, avec une seule légende.
% Un graphique est en haut, l'autre en bas.
% figdhb: FIGure Double, Haut-Bas.
%-----------------------------------------------------------------------
%
\newcommand{\figdhb}[7]
	{
	\begin{figure}[htbp]
		\centerline{
			\includegraphics
				[angle=#4, 
				width=#3, 
				keepaspectratio=true,
				clip=true]
				{#1}
		}
		\centerline{
			\includegraphics
				[angle=#4, 
				width=#3, 
				keepaspectratio=true,
				clip=true]
				{#2}
		}
		\caption{ {\bf #5} {\em #6} {#7} }
		\label{#1}
	\end{figure}
	}
%
%-----------------------------------------------------------------------
% Image graphicx double. I.e. lit 2 fichiers EPS en entrée,
% et crée une seule figure, avec une seule légende.
% Les 2 graphiques sont côte-à-côte.
% figdcc: FIGure Double, Côte à Côte.
%-----------------------------------------------------------------------
%
\newcommand{\figdcc}[7]
	{
	\begin{figure}[htbp]
		\centerline{
			\includegraphics
				[angle=#4, 
				width=#3, 
				keepaspectratio=true,
				clip=true]
				{#1}
			\includegraphics
				[angle=#4, 
				width=#3, 
				keepaspectratio=true,
				clip=true]
				{#2}
		}
		\caption{ {\bf #5} {\em #6} {#7} }
		\label{#1}
	\end{figure}
	}
%
%-----------------------------------------------------------------------
% Commande vide "£": elle sert seulement à définir une fonction
% qui ne fait rien sous latex, mais sera reconnue comme
% une chaîne de caractères par le coloriseur du source TEX.
% But: rendre des sources TEX plus lisibles, en pouvant coloriser
% des zones de texte particulières.
%-----------------------------------------------------------------------
%
%\newcommand{\£}{}
